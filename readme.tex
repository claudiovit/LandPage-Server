Strapi V4 - Paginação e Populate (Atualização 30/05/2022)
Na versão atual do Strapi (v4) o formato da API é completamente diferente.

URL da API
Um ponto importante é adicionar "api" na URL. Por exemplo, ao invés de:

http://localhost:1337/pages
Adicione "api":

http://localhost:1337/api/pages
Populate (Plugin)
Outro ponto importante é que Dynamic Zones e componentes não são populados automaticamente, por isso teremos que informar ao Strapi como fazer isso.

Eu vou usar um plugin chamado "strapi-plugin-populate-deep". Isso vai deixar as coisas muito mais simples. Simplesmente instale o plugin:

npm install strapi-plugin-populate-deep
E nas URLs onde tu quiser popular os valores, adicione o seguinte:

http://localhost:1337/api/pages?populate=deep
Isso garantirá que todas as dynamic zones e components serão populados com valores.

Se quiser algo mais fragmentado, leia a seguir.

Populate (nativo) e Paginação
Para algo nativo do Strapi, podemos fazer via código ou via URL. Veja como fazer "populate" via URL.

Por exemplo, para popular nossas "sections", use:

http://localhost:1337/api/pages?populate[sections][populate]=*
Para paginação, use pagination[page] e pagination[pageSize]:

http://localhost:1337/api/pages?populate[sections][populate]=*&pagination[page]=1&pagination[pageSize]=10
A URL acima está na página 1, o número de elementos por página é 10.

Para ordenação, use sort[índice]=campo:asc ou desc. Veja:

http://localhost:1337/api/pages?populate[sections][populate]=*&pagination[page]=1&pagination[pageSize]=1&sort[0]=id:desc
A URL acima usa o índice 0 (primeiro campo usado para ordenação), o campo é o ID, o formato é decrescente (sort[0]=id:desc).

Minha URL final para obter os dados
Como vou usar o plugin indicado anteriormente, minha URL vai ficar bem simples para popular os dados. Eu só preciso alterar a paginação. Veja:

http://localhost:1337/api/pages?populate=deep&pagination[pageSize]=1&sort[0]=id:desc
Caso queira seguir como estou fazendo, a URL acima é a mesma que estou usando daqui em diante.

Leia mais na documentação
Mais dados sobre isso em: https://docs.strapi.io/developer-docs/latest/developer-resources/database-apis-reference/rest-api.html

Outras atualizações virão ao longo das aulas. Agradeço a compreensão.
